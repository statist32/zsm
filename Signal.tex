\documentclass{article}
\usepackage{amsmath}
\usepackage{bm}
\usepackage[utf8]{inputenc}
\begin{document}

\section{Energie}

\subsection{Energie/Leistung}
\begin{tabular}{c c c }
Energiesignal & $E_X = \int_{-\infty}^\infty |x(t)|^2 dt$ &$0<E_x <\infty \Rightarrow P_X=0$ \\
&$E_X = \sum_{-\infty}^\infty |x(t)|^2 dt$ & \\
Leistungssignal & $P_X = \lim\limits_{ T \to \infty}\frac{1}{2T} \int_{-T}^T |x(t)|^2dt $ &  $0<P_X<\infty \Rightarrow E_X \rightarrow \infty$ \\
 &$P_X = \lim\limits_{ T \to \infty}\frac{1}{2K+1} \sum_{-K}^K |x[k]|^2 $ & \\
\end{tabular}

\begin{itemize}
\item reell existieren keine Leistungssignale
\item periodisches Signal $\Rightarrow$Leistungssignale
\item endlicher Flächeninhalt $\Rightarrow$ Energiesignal
\end{itemize}

\subsection{Signalzerlegung}
\begin{itemize}
\item $f_g(t)= \frac{1}{2}(f(t)+f(-t))$
\item $f_u(t)= \frac{1}{2}(f(t)-f(-t))$
\end{itemize}

$$
\Gamma(t)=
\begin{cases}
1, \; t > 0 \\
\frac{1}{2}, \; t= 0 \\
0, \;sonst
\end{cases}
$$

$$
rect(\frac{t}{T})=
\begin{cases}
1, \; t \in  [-\frac{T}{2}, \frac{T}{2}] \\

0, \;sonst
\end{cases}
$$

$$
\Lambda(\frac{t}{T})=
\begin{cases}
-T+|t|, \;[-T, 0] \\
1-t, \;  [0,T] \\
0, \;sonst
\end{cases}
$$

$$
\delta(t)=
\begin{cases}
1, \;t = 0\\
0, \;sonst
\end{cases}
$$
\begin{itemize}
\item Ausblendeigenschaft $f(t)\cdot\delta(t-t_0) = f(t_0)\cdot\delta(t-t_0)$ \\  $ \int_{-\infty}^\infty f(t)\cdot \delta(t-t_0) dt = f(t_0)$
\item $\delta(bt) = \lim\limits_{\varepsilon \to 0} \frac{1}{\varepsilon}$ rect$(\frac{bt}{\varepsilon})$
\end{itemize}

$$sinc(t) = \frac{sin(t)}{\pi t}$$
$$A\cdot rect(\frac{t-t_0}{T})$$

\begin{itemize}
\item $A$: Skalierung
\item $t_0$: Verschiebung
\item $T$: Dehnung/Stauchung
\end{itemize}


\section{LTI-Systeme}
\subsection{Linearität}
\begin{itemize}
\item Additivität: $S\{x_1(t) + x_2(t)\} = S\{x_1(t)\} + S\{x_2(t)\}$
\item Homogenität: $\{ a\cdot x(t)\} = a S\{x(t)\} = a\cdot y(t)$
\end{itemize}


\subsection{Zeitinvarianz}
\begin{itemize}
\item $y(t) = S\{x(t)\} \Rightarrow y(t-\tau) = S\{x(t-\tau)\} = S\{\tilde{x}(t)\}$
\item $-\tau$ in das Argument von x $x(3t^2-\tau)$ 
\end{itemize}

\subsection{Kausalität}
\begin{itemize}
\item Kausal $\Leftrightarrow h(t) = 0 \;  \forall t < 0$
\item $h[k] = 0 \; \forall k<0$
\end{itemize}

\subsection{BIBO}
\begin{itemize}
\item $\int_{-\infty}^\infty |h(t)|dt < \infty$
\item $\sum_{-\infty}^\infty |h[k]| < \infty$
\item also absolut integrierbar/summierbar
\end{itemize}

\subsection{Superpositionsprinzip}
\begin{itemize}
\item $x(t) := \sum_n a_n\cdot e^{s_nt}$
\item $y(t) = \sum_n  a_n \cdot e^{s_n t} \cdot H(S_n)$
\end{itemize}


\subsection{zeitkontinuierliche Faltung}
\begin{itemize}
\item $y(t) = x(t)*h(t) = \int_{-\infty}^\infty x(\tau)h(t-\tau) d\tau$
\item $x(a(t+T))*\delta(t-t_0) = x(a(t+T-t_0)$ (t im Arg(x) durchArg(Dirac) ersetzen)
\item kommutativ: $x(t)*h(t) = h(t)*x(t)$
\item distributiv: $x(t)*(h_1(t)+h_2(t)) =x(t)*h_1(t)+x(t)*h_2(t)$
\item assoziativ: $x(t)*(h_1(t)*h_2(t)) = (x(t)*(h_1(t))*h_2(t)$
\end{itemize}



\section{Foruiertreihen}
\begin{itemize}
\item $x(t) = \frac{a_0}{2} + \sum_{n=1}^\infty a_n cos(\omega_0nt)+b_nsin(\omega_0nt)$
\end{itemize}
\begin{tabular}{ c c}
 Gleichanteil & $\frac{a_0}{2}$\\
 Koeff. Grundschw. & $a_1,b_1$ \\
 Koeff. Oberschw. & $a_n, b_n$ \\
 \end{tabular}

\subsection{Komplexe Darstellung}
\begin{itemize}
\item $x(t) = \sum_{n = - \infty}^\infty X_ne^{jn\omega_0t} \; \omega_0 = \frac{2\pi}{T}$
\item $X_n = \frac{1}{T} \int_0^T X(t) e^{-jn\omega_0t}dt$
\item $|x(t)|^2 = x(t)\cdot x(t)^*$
\end{itemize}
\begin{tabular}{c  c c}
$a_0$ &=& $2X_0$ \\
 $a_n$&= & $X_n +X_{-n}$ \\
 $b_n$&= & $j(X_n-X_{-n})$ \\
 $X_0$&= & $\frac{a_0}{2}$ \\
 $X_n$ &=& $\frac{1}{2}(an_-jb_n)$ \\
 $X_{-n}$ &=& $\frac{1}{2}(a_n+jb_n)$ \\
 \end{tabular}


\subsection{Eigenschaften}
\begin{tabular}{c c c}
$ax(t) + by(t)$ &$\leftrightarrow$& $aX_n+bY_n$ \\
$x(t-t_0)$ &$\leftrightarrow$& $X_n\cdot e^{-j\omega_0t_0}$ \\
$x(-t)$ &$\leftrightarrow$& $X_{-n}$ \\
$x(at)$ &$\leftrightarrow$& $X_n$ mit Per. $\frac{T}{a}$, $a > 0$ \\
$x(t)\cdot y(t)$ &$\leftrightarrow$& $x(t)*y(t)$ \\
\end{tabular}

\subsection{Dirichlet-Kritirien}
\begin{itemize}
\item $\int_T |x(t)|dt \leq \infty$
\item endlich viele Minima und Maxima in einer Periode
\item endliche Anzahl der Unstetigkeitsstellen in einer Periode
\item $\Rightarrow$ garantieren punktweise Konvergenz, außer an Unstetigkeitsstellen
\item An Unstetigkeitsstellen gegen Mittelwert des linken und rechten Grenzwerts
\end{itemize}



\subsection{FR und FT}
\begin{tabular}{ c| c| c }
& kontinuierlich & periodisch \\
FR & & \\ \hline
$x(t)$ & ja & ja \\
 $X_n$ &nein& ja\\
 FT & & \\ \hline
 $x(t)$ & ja & nein \\
 $X^F(\omega)$ & ja & nein \\
\end{tabular}

\subsection{Fourier Transformation}
\begin{itemize}
\item $F\{x(t)\} = X^F(\omega) = \int_{-\infty}^\infty x(t) e^{-j\omega t} dt \leftrightarrow x(t)$
\item $F^{-1}\{X^F(\omega)\} = x(t) = \frac{1}{2\pi}\int_{-\infty}^\infty X^F(\omega) e^{j\omega t}$
\end{itemize}

\subsection{Eigenschaften}
\begin{tabular}{ c c c}
$ax(t)+by(t)$ & $\leftrightarrow$ & $aX^F(\omega)+bY^F(\omega)$\\
$x(t-t_0)$ & $\leftrightarrow$ & $X^F(\omega)e^{-j\omega t_0}$\\
$x(t)e^{j\omega_0t}$ & $\leftrightarrow$ & $X^F(\omega-\omega_0)$\\
$x(at)$ & $\leftrightarrow$ & $\frac{1}{|a|}F^F(\frac{\omega}{a})$\\
$\dot{x}(t)$ & $\leftrightarrow$ & $ jw X^F(\omega)$\\
$x(t)\cdot y(t)$ & $\leftrightarrow$ & $\frac{1}{2\pi}X^F(\omega)*Y^F(\omega)$\\
$x^*(t)$ & $\leftrightarrow$ & $X^{F*}(-\omega)$\\
$$ & $\leftrightarrow$ & $$\\
$$ & $\leftrightarrow$ & $$\\
\end{tabular}

\subsection{FT Paare}
\begin{tabular}{ ccc}
$\delta(t)$ & $\leftrightarrow$ & $1$\\
$rect(\frac{t}{T})$ & $\leftrightarrow$ & $|T|sinc(\frac{T\omega}{2\pi})$\\
$\Lambda(\frac{t}{T})$ & $\leftrightarrow$ & $|T|sinc(\frac{T\omega}{2\pi})$\\
$sin(\omega_0 t)$ & $\leftrightarrow$ & $j\pi[\delta(\omega+\omega_0)-\delta(\omega-\omega_0)]$\\
$cos(\omega_0t)$ & $\leftrightarrow$ & $\pi[\delta(\omega+\omega_0)-\delta(\omega+\omega_0)]$\\
$$ & $\leftrightarrow$ & $$\\
$$ & $\leftrightarrow$ & $$\\

\end{tabular}

\subsection{}
\begin{itemize}
\item
\item
\end{itemize}




\subsection{Additionstheoreme}
\begin{itemize}
\item $sin(a)sin(b)=\frac{1}{2} (cos(a-b)-cos(a+b))$
\item $cos(a)cos(b)=\frac{1}{2} (cos(a-b)+cos(a+b))$
\item $sin(a)cos(b)=\frac{1}{2} (sin(a-b)-sin(a+b))$
\item $cos(\omega_0 t) = \frac{1}{2}(e^{j\omega_0 t}$
\end{itemize}


\subsection{}
\begin{itemize}
\item
\item
\end{itemize}





\end{document}