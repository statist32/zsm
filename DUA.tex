\documentclass{article}
\begin{document}
\section{Invariante}
\begin{itemize}
\item Vorm ersten Durchlauf erfüllt sein (Initialisierung, Induktionsanfang)
\item Muss bei jedem Schleifendurchlauf erhalten bleiben (Erhaltng, Induktionsschritt)
\item Invar nach Beendigung der Schleife zeigt Korrektheit (Terminierung)
\end{itemize}


\subsection{Rekursionsgleichungen}
\begin{itemize}
\item ineinander einsetzen
\item Summe erkennen und zusammengefasst aufschreiben
\end{itemize}
\subsection{Mastertheorem}
\subsubsection{Additives Mastertheorem}
\begin{itemize}
\item $ a,b,c \; positiv \; n= b^k$
\item $T (n) \leq \left\{
\begin{array}{ll}
c & n  = 1 \\
a\cdot T(\frac{n}{b}) +c & \, n>1 \\
\end{array}
\right. $
\end{itemize}
\begin{tabular}{l l  l}

$T(n) \leq c\cdot \frac{a}{a-1}n^{log_b(a)} - \frac{c}{a-1}  $&$= O(n^{log_b(a)}) $&$ \; falls \; a>1$ \\
$T(n) \leq c\cdot \frac{a}{a-1}n - \frac{c}{a-1} $&$ = O(n)  $& $\; falls \; a=b>1$\\
$T(n) \leq c\cdot log_b(n) +c  $&$= O(log(n)) $&$\; falls \; a=1$\\
\end{tabular}
\subsubsection {allgemeines Mastertheorem}
\begin{itemize}
\item $a,b,d,q \geq  1$
\item $T (n) \leq \left\{
\begin{array}{ll}
d & n\leq q \\
a\cdot T(\frac{n}{b}) +f(n) & \, n>q \\
\end{array}
\right. $
\end{itemize}
\begin{tabular}{l l}
$f(n) = O(n^{log_b(a)-\epsilon}) \; mit \; \epsilon > 0$ & $T(n) = O(n^{log_b(a)}) $\\
$f(n) = \Theta(n^{log_b(a)}) $ & $T(n) = O(n^{log_b(a)} log(n)) $\\
$f(n) = \Omega(n^{log_b(a)+\epsilon}) \; mit \; \epsilon > 0,$ & $T(n) = \Theta(f(n)) $\\
$a\cdot f(\frac{n}{b}) \leq \delta  \cdot f(n), \delta < 1, n \to \infty$ & \\
\end{tabular}

\section{ O-Notation}
\subsection{Definition}
\begin{itemize}
\item $f = O(g) \Leftrightarrow  \exists  c> 0 \exists n_0 \in N \; \forall n \geq n_0 : f(n) \leq c g(n)$ \\ (f wächst asymptotisch höchstens so schnell wie g)
\item $f = \Omega(g) \Leftrightarrow g = O(f)$ \newline (f wachst asymptotisch mindestens so schnell wie g)

\item f = $\Theta (g) \Leftrightarrow f = O(g) \; und \; g = O(f)$ \newline
(f und g wachsen asymptotisch gleich schnell)
\item $f = o(g) \Leftrightarrow \forall c > 0 \; \exists n_0 \in N \forall n\geq n_0 : f(n) < c g(n)$ \\ 
$ \lim\limits_{n \to \infty}\frac{f(n)}{g(n)} = 0 $\newline (f wächst asymptotisch langsamer als g)
\item $f = \omega(g) \Leftrightarrow g = o(f)$ \newline (f wächst asymptotisch langsamer als g)
\end{itemize}
\subsection{Eigenschaften}
\begin{itemize}
\item $f = o(g) \Rightarrow f = O(g)$
\item $f = o(g) \; und \;  h = o(g) \Rightarrow f+h = o(g) (auch O, \Omega, \omega, \Theta)$
\item $f = o(g) \; und \;  h = o(g) \Rightarrow f\cdot h = o(g^2) (auch O, \Omega, \omega, \Theta)$
\end{itemize}
\subsection{Reihenfolge}
$c < log(n) < n^{\frac{1}{k}} < n < n log(n) < n^2 < n^k < 2^n$
\section{Sortieralgorithmen}
\subsection {Bubblesort}
\begin{itemize}
\item Jeden Durchlauf wird das größte Element auf die n-ite Stelle getauscht. Jeder Vergleich ggf ein Swap.
\item inkrementelle, inplace, stabil
\item $BC: \Theta(n)\; AV: \Theta(n^2) \; WC: \Theta(n^2)$
\end{itemize}

\subsection{Insertionsort}
\begin{itemize}
\item Key wird in sortiertes Array eingeordnet. Key wird gemerkt, falls kleiner wird das gö\ss ere Element auf Pos von Key kopiert aber Key wird erst kopiert, wenn die richtige Stelle gefunden worden ist.
\item inkrementelle, inplace, stabil
\item $BC: \Theta(n)\; AV: \Theta(n^2) \; WC: \Theta(n^2)$ 
\end{itemize}

\subsection{Mergesort}
\begin{itemize}
\item Teile Array bis auf ein Element Array und sortiere beim rekursiven zusammenfügen. Geteilt wird p bis q, q+1 bis r. \\
Merge vergleicht erste Pos von den Arrays und fügt immer das kleinere ins Zielarray ein.
\item D\&C, 
\item $BC: \Theta(nlog(n))\; AV:\Theta(nlog(n))\; WC: \Theta(nlog(n))\;$
\end{itemize}

\subsection{Quicksort(A,p,r)}
\begin{itemize}
\item Teile Array nach Pivotelement und füge Pivot dann an Grenze ein. Dann partition bis q-1 und 
\item D\&C, 
\item $BC: \Theta(nlog(n))\; AV:\Theta(nlog(n))\; WC: \Theta(n^2)\;$
\end{itemize}

\section{Heap}
\subsection{Definition}

\begin{itemize}
\item $ Heap := A[i] \geq A[2i] und A[i] \geq A[2i+1]$ \\
\item jeder Knoten hat mindestens so großen Wert wie seine Kinder \\
\item Wird als binärer Baum dargestellt. \newline
\end{itemize}

\subsection{Eigenschafen}
\begin{itemize}
\item Baumtiefe $\lfloor log(n)\rfloor$
\item $A[\lfloor \frac{n}{2} \rfloor +1] \; bis \; A[n]$ sind Kinder
\item Maximum ist die Wurzel A[1]
\end{itemize}
\subsection{Heapify}
\begin{itemize}
\item Tausche mit größtem Kind
\item wird auf getauschten Knoten erneu aufgerufen bis er richtit steht
\end{itemize}
\subsection{Build-Heap}
\begin{itemize}
\item Bei uns meist MaxHeap
\item Heapify auf $A[\lfloor \frac{n}{2} \rfloor] \; downto  \;A[1]$
\item $BC: \Theta(n)\; AV:\Theta(n)\; WC: \Theta(n)\;$
\item \#Vertauschungen $\leq \sum_{i=1}^{\lfloor \frac{n}{2} \rfloor} (log(n) -log(i))$
\item \#Vergleiche $\leq$ \#Vertauschungen
\end{itemize}

\subsection{Heap-Sort}
\begin{itemize}
\item Build-Heap
\item Sort-Heap \\
$Vertausche A[1] und A[i]$\\
Danach Heapify(A[i ... i-1])\\
\item $BC: \Theta(nlog(n))\; AV:\Theta(nlog(n))\; WC: \Theta(nlog(n))\;$
\end{itemize}

\subsection{Bucket-Sort}
\begin{itemize}
\item Hänge $a_j$ an Liste L[$a_j$] an. Gebe alle Listen aus
\item stabil
\item Zahlen aus \{1,...,m\} und O(n+m) und braucht O(n+m) Platz
\end{itemize}

\section{Dynamische Arrays}
\begin{itemize}
\item Falls Array A nicht mehr ausreicht (n>w). Verdoppel Array 
\item Falls $\frac{1}{4} and n>0$ Halbiere Array
\end{itemize}

\section{Stack}
\begin{itemize}
\item Empty(S), Pop(S), Push(S), Top(S)(Pos)
\end{itemize}

\section{Queue}
\begin{itemize}
\item Enqueue,Dequeue, Head(Q), Tail(Q)(erste freie Pos)
\item falls Array vonn starte bei 1, falls frei
\end{itemize}

\section{Doppelt verkettete Listen}
\begin{itemize}
\item Head(L), Insert(L,x)(hängt vorne dran), Remove(L,x)
\item key(x), next(x), prev(x)
\item falls next(x) = nil, x letztes Element
\end{itemize}

\section{Skiplisten}
\begin{itemize}
\item verschiedene Niveaus 
\item perfekte hat $ \lceil  log(n)\rceil$ Niveaus (ermöglicht binäre Suche)
\item left(v), right(v), down(v), up(v), Search(L,x), Insert(L,v) (mit RandomHight)
\item Niveau 0, wo jeder Knoten ist ($2^0$)
\item Niveau k behinaltet jeden $2^k$-ten Knoten
\end{itemize}

\section{Binäre Suchbäume}
\subsection{Binäre Suchbäume}
\begin{itemize}
\item lc[x], rc[x], p[x],root[x]
\item Inorder-Tree-Walk(x) gebe Knoten sortiert, nach abgeben, aus. \\ Gehe am Ende von ganz rechts zur Wurzel zurück. $\Theta(n)$
\item Baumsuche(x,k) meist mit x = root[T]
\item Min/Max linkestes/rechtestes Element
\item Nachfolger(x) linkestes Element im rechten Teilbaum. Wenn nicht verfügbar im Baum aufsteigen
\item Delete(x), x hat 2 Kinder, Nachfolger von x wird verschoben, ggf wiederholen
\end{itemize}

\subsection{Balancierte Suchbäume}
\begin{itemize}
\item Höhe höchstens $2log(n+1)-2$
\item Rotation ändern nur Höhe
\item Rechtsrota(T,x),Linksrota(T,x)
\item Balance(t): \\Falls lc[t] und rc[lc[t]] größer dann Linksrota(lc[t]), sonst Rechtsrota(t) \\
	Falls rc[t] und lc[rc[t]] größer dann Rechtsrota(rc[t]), sonst Linksrota(t)
\end{itemize}

\section{Hashing}
\subsection{Geschlossene Adressierung}
\begin{itemize}
\item Kollisionsuaflösung durch Listen
\item Suchen/Löschen AV: $\Theta(1+\alpha)$, Falls  m= $\Theta(n)$ $.\Theta(1)$
\end{itemize}


\subsection{Offene Adressierung}
\begin{itemize}
\item (nächste) freie Stelle in der Hashtabelle
\item Lineares Hashing: $h(k,i) = (h'(k) +i) mod \; m$
\item Quadratisches Hashing: $h(k,i) = (h'(k) +c_1i+ c_2 i^2) mod \; m$
\item Delete problematisch, deshalb offene Adressierung bei Anwendungen ohne Delete nutzen
\item falls zu viele deleted, dann neu hashen in größerer Tabelle(amortisierte Laufzeit)
\end{itemize}

\subsection{Kuckuckshashing}
\begin{itemize}
\item 2 Hashfunktionen mit eine Tabelle
\item Insert: Falls belegt, neuen Wert einfügen und alten mit anderer Funktion neu hashen
\item max $d log(n)$ Hashversuche
\end{itemize}


\section{Graphentheorie}
\subsection{Adjazenzmatrix/liste}
\begin{itemize}
\item Zeile: von, Spalte: nach
\item Zeile: von, Einträge: erreichende Knoten (einfach verkettete liste)
\end{itemize}

\subsection{SSSP}
\subsubsection{BFS}
\begin{itemize}
\item berechnet Abstand von allen Knoten zu s in einem ungewichteten Graphen
\item d[u] ist der Abstand am Anfang $\infty$
\item $\pi$[u] ist der Vorgänger am Anfang NIL
\item Color[u] = weiß, grau,schwarz am Anfang weiß
\item Queue zum Speichern der grauen Knoten
\item BFS entdeckt alle Knoten $v\in V$, die von s aus erreichbar sind
\item entdeckt die Zusammenhangskomponenten
\item $O(|V|+|E|)$
\end{itemize}

\subsubsection{DFS}
\begin{itemize}
\item Neue Konten vom zuletzt gefundenen Knoten entdeckt. Fals alle adj[v] entdeckt gehe zu $\pi$[v]
\item löst nicht SSSP, bildet Spannbaum
\item d[u] ist der Abstand am Anfang $\infty$
\item $\pi$[u] ist der Vorgänger am Anfang NIL
\item Color[u] = weiß, grau,schwarz am Anfang weiß
\item Stack zum Speichern der grauen Knoten
\item Baumkante: rot, Rückkante: grün (auf grauen), Sonstigekante: blau (auf schwarzen)
\item G enthält einen Kreis $\Leftrightarrow$ DFS(G) erzeugt mindestens eine Rückwärtskante
\item DAG $\Leftrightarrow$ es existiert eine topologische Sortierung (alle Kanten einzeichnen)
\item TopoligischesSorrtieren(G) legt eine Liste an und fügt die schwarzen Knoten vorne an
\end{itemize}

\subsection{Dijkstra}
\begin{itemize}
\item kein negativer Kreis aber gewichtet
\item setze alle Knoten auf $\infty$ und update alle adj(u) mit dem Gewicht, dalls geringer als aktueller Wert
\item Laufzeit mit Heaps:$O((|V|+|E|)log(|V|))$
\end{itemize}


\section{Dynamische Programmierung}
\begin{itemize}
\item Berechne die Funktionswerte iterativ und bottom-up und nutze Teilergebnisse
\end{itemize}
\subsection{Bellman-Ford}
\begin{itemize}
\item ohne negative Zyklen erhält man kürzesten Pfad
\item  Laufzeit $O(|V|^2+|V||E|)$ und Speicherbedarf $O(|V|^2)$
\end{itemize}
\subsection{APSP}
\subsubsection{Floyd-Warshall(W,n)}
\begin{itemize}
\item Keine negative Zyklen
\item Initialisiere Adjazenzmatrix mit 1 Nachbarknoten
\item halte k mal Spalte K und Zeit K fest und guck, ob geringere Summen entstehen
\item Laufzeit $O(|V|^3)$ mit Platz $O(|V|^2)$
\end{itemize}



\section{Greedy}
\subsection{Allgemein}
\begin{itemize}
\item das zum Zeitpunkte beste auswählen, was man machen kann
\item EDF für Scheduling Probleme
\end{itemize}
 
\subsection{Kruskal}
\begin{itemize}
\item gewichteter, ungerichteter, zusammenhängender Graph errechnet daraus einen MST
\item Nimm Kante mit geringstem Gewicht, die zwei Bäume im aktuellen aufspannenden Wald verbindet und füge diese zu A hinzu
\item rote Kante: Spannbaum, grüne Kante: ungebraucht
\end{itemize}

\subsection{Union-Find - SIEHE ÜBUNG}
\begin{itemize}
\item Jede Menge ist eine lineare Liste, die durch das erste Element repräsentiert wird
\item doppelt verkettete Liste mit einem weiteren Zeiger auf den Kopf
\item hänge kürzere Liste an längere Liste
\end{itemize}

\subsection{Prim}
\begin{itemize}
\item Lässt einen Wald wachsen. Immer nur von bekannten Knoten aus auswählen
\item  Laufzeit $O(|E| log(|V|)$
\end{itemize}







\subsection{}
\begin{itemize}
\item
\end{itemize}



\section{Rechentricks}
\begin{itemize}
\item Arithmetische Reihe $\sum_{i=1}^n k  = \frac{n(n+1)}{2} $
\item logarithmus!!! bsp laufzeit aus probe 1
\item Stirling'sche Formel $n! \approx (\frac{n}{e})^n \cdot \sqrt{2\pi n}$ \newline
$ \Rightarrow  log(n!) \approx nlog(n)-nlog(e)$
\item $log((\frac{P}{Q})^k) = k \cdot log(\frac{P}{Q}) = k \cdot (log(P) - log(Q)) $
\item $log_a(P) = \frac{log_b(P)}{log_b(a)}$

\item Union Find

\end{itemize}

\end{document}