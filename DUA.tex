\documentclass{article}
\begin{document}
\section{Invariante}
\begin{itemize}
\item Vorm ersten Durchlauf erfüllt sein (Initialisierung, Induktionsanfang)
\item Muss bei jedem Schleifendurchlauf erhalten bleiben (Erhaltng, Induktionsschritt)
\item Invar nach Beendigung der Schleife zeigt Korrektheit (Terminierung)
\end{itemize}

\section{Divide \& Conquer}
\begin{itemize}
\item Aufteilen und Lösen
\end{itemize}


\subsection{Rekursionsgleichungen}
\begin{itemize}
\item ineinander einsetzen
\item Summe erkennen und zusammengefasst aufschreiben
\end{itemize}

\section{ O-Notation}
\subsection{Definition}
\begin{itemize}
\item $f = O(h) \Leftrightarrow  \exists  c> 0 \exists n_0 \in N \forall n \geq n_0 : f(n) \leq c g(n)$ \newline (f wächst asymptotisch höchstens so schnell wie g)
\item $f = \Omega(g) \Leftrightarrow g = O(f)$ \newline (f wächst asymptotisch mindestens so schnell wie g)

\item f = \Theta(g) \Leftrightarrow f = O(g) und g = O(f) \newline
(f und g wachsen asymptotisch gleich schnell)
\item f = o(g) \Leftrightarrow \forall c > 0 \exists n_0 \in N \forall n\beq n_0 : f(n) < c g(n)
\end{itemize}



\section{Rechentricks}
\begin{itemize}
\item Arithmetische Reihe $\sum_{i=1}^n k  = \frac{n(n+1)}{2} $
\end{itemize}

\end{document}