\documentclass{article}

\begin{document}
\section{Kombinatorik}
\begin{tabular}{c| c |c}
  & ohne & Zurück legen \\ \hline
  Reigenfolge & $\frac{n!}{(n-k)!}$&$n^k$ \\
  ohne&${n \choose k}$ & ${n + k -1 \choose k}$ \\
  
\end{tabular}

\section{Diskrete Zufallsvariablen}
\subsection{Axiomatscher Ansatz}
\begin{itemize}
\item $P(E)\geq 0$
\item $P(\Omega) = 1$
\item $P(E\cup F) = P(E)+P(F), \; wenn\;  E\cap F = \emptyset $
\item $\Omega = E \cup \overline{E}$
\end{itemize}


\subsection{Gemeinsame und bedingte Wahrscheinlichkeiten}
\begin{itemize}
\item$P(EF) = P(E\cap F)$
\item $P(E|F) = \frac{P(EF)}{P(F)}$
\item $P(EF) = P(E|F)P(F)$
\end{itemize}




\subsection{statistische Unabhängigkeit}
\begin{itemize}
\item $P(EF) = P(E) P(F)$
\end{itemize}

\subsection{Bayes-Regel}
\begin{itemize}
\item $P(F|E) = \frac{P(E|F) P(F)}{P(E|F) P(F)+P(E|\overline{F})P(\overline{F})}$
\end{itemize}


\subsection{Gesetz der Gesamtwsk}
\begin{itemize}
\item $P(F) = \sum_{i=1}^K P(F|E_i)P(E_i)$
\end{itemize}



\subsection{Diskrete Zufallsbvariablen}

\begin{itemize}
\item Eine Zufallsvaribale ZV $X(\omega)$ ist eine Funktion, die jedem möglichen \newline Ergebnis $\omega \in \Omega$ eines Zufallsexperimentes eine reelle Zahl $X(\omega)$ zuordnet.
\item Jede reelle Zahl $X(\omega)$ hate eine zugehörige Wahrscheinlichkeit.\newline Wir schreiben diese als $P(X(\omega) = x) = P_X(x)$
\item $P(4\leq X \leq 6) = P(X\leq 6) -P(X \leq 3)$
\end{itemize}

\subsection{Wahrscheinlichkeitsverteilung}
\begin{itemize}
\item $P_X(x) \hat{=} P(X=x)$
\item $ 0 \leq P_X(x) \leq 1$
\item $\sum_X P_X(x) = 1$
\end{itemize}

\subsection{Poisson-ZV}
\begin{itemize}
\item $P_X(x) = \frac{\lambda^x e^{-\lambda}}{x!}$
\item
\end{itemize}


\subsection{Zwei diskrete ZV}
\begin{tabular}{c | c}
WskVerteilung & Formel \\ \hline
Gemeinsame & $P_{XY}(x,y) \hat{=} P(X=x,Y=y)$\\
Bedingte& $P_{X|Y}(x|y) = \frac{P_{XY}(x,y)}{P_Y(y)}$\\
Rand & $P_X(x,y) = \sum_Y P_{XY}(x,y)$\\

\end{tabular}

\subsection{binomial Verteilung}
\begin{itemize}
\item $P_X(k) = {n \choose k} p^k (1-p)^{n-k} = \frac{n!}{k! (n-k)!} p^k (1-p)^{n-k} $
\end{itemize}

\subsection{Erwartungswerte und Momente}
\begin{tabular}{c|c}
& Formel \\ \hline
Durchschnitt & $\mu = \frac{1}{n}\sum_{i=1}^L g_i N_i$ \\
quadratische Abw. & $\sigma ^2 = \frac{1}{n} \sum_{i=1}^L (g_i-\mu)^2N_i$ \\
Mittelwert & $\mu_X = E\{X\} \hat{=} \sum_X xP_X(x)$ \\
Varianz & $var(X) \hat{=} E{(X-\mu_X)^2} = \sum_X (x-\mu_X)^2P_X(x)$\\
Standard Abw. & $\sigma_X = \sqrt{var(X)}$\\
Korrelation & $\mu_{XY} = E\{XY\} = \sum_X \sum_Y xyP_{XY}(x,y)$\\
Kovarianz & $\sigma_{XY} = E\{(X-\mu_X)(Y-\mu_Y)\} =\sum_X \sum_Y (x-\mu_X)(y-\mu_Y)P_{XY}(x,y) $\\
Linearität & $E\{aX+bY\} = aE\{X\} + bE\{Y\}$ \\
\end{tabular}
\subsection{Unkorreliertheit}
\begin{itemize}
\item $E\{XY\} = E\{X\}E\{Y\}$, d.h $ \sigma_{XY} = 0$, dann sind X und Y unkorreliert
\item statistische Unabhängigkeit bedeutet Unkorreliertheit aber \textbf{nicht umgekehrt}
\item Korrelationskoeffizient $\rho_{XY} = \frac{\sigma_{XY}}{\sigma_X \sigma_Y}$
\item $-1\leq  \rho_{XY} \leq  1$
\end{itemize}

\subsection{Lineare Schätzung}
\begin{itemize}
\item opt Schätzer, der MSE minimiert: \newline $\hat{Y} = \mu_Y + \rho_{XY}\frac{\sigma_Y}{\sigma_X} (X-\mu_x) $
\item dessen MSE = $\sigma_Y^2(1-\rho_{XY}^2)$
\end{itemize}

\subsection{}
\begin{itemize}
\item Stichprobenmw: $\hat{\mu}_n = \frac{1}{n} \sum_{i=1}^n x_i$
\item Stichprobenvar: $\hat{\sigma}_n^2 = \frac{1}{n-1} \sum_{i=1}^n(x_i-\hat{\mu}_n)^2$
\end{itemize}


\subsection{Zeitdiskrete Markoff-Ketten}
\subsubsection{Definition}

Eine Folge von ganzwertigen ZVn. \newline
Ereigniss = Zustand \newline
Zeithomogen, wenn keine Abhängikeit von der Zeit besteht \newline
$P(X_{n+1} = j | X_n = i) sind Übergangswahrscheinlichkeiten$ \newline
$P(X_n = i)$ sind Zustandwahrscheinlichkeiten \newline
$P(X_{n+1} = | X_n = i_n , ..., X_0 = i_0) = P(X_{n+1} = i_{i+1} | X_n = i_n)$


\subsubsection{station{\"a}re Verteilung}
\begin{itemize}
\item $q_n = q_0 P^n$
\item Station{\"a}re Verteilung, wenn $\pi = \pi P$ 
\item $\pi = \pi P \Leftrightarrow \pi(P - I) = 0$
\item irreduzibel = Jeder Zusatand erreichbar
\item irreduzibel $\Leftrightarrow \leq $ 1 stationäre Verteilung
\item Eine irreduzible, aperiodische Kette mit einer endlichen Anzahl von Zuständen hat eine eindeutige stationäre Verteilung.
\end{itemize}

\subsubsection{Berechnung der stationären Verteilung}


\subsection{Stetige Zufallsvariablen}
\subsubsection{Wkdichte- und kumulative Verteilungsfunktionen}
kummulative Verteilungsfunktion $F_X (x) \hat{=}  P(X \leq x ) = \int_{-\infty}^x f_x(\xi)d\xi $ \newline
Wahrscheinlichkeitsdichtefunktion $f_X(x) = \frac{d}{dx} F_X(x)$

\paragraph{Eigenschaften}
\begin{itemize}
\item $P(a < X \leq b) = P(X\le b) - P(X \le a) = F_X (b) - F_X(a)$
\item $P(X > a) = 1-P(X \le a) = 1- F_X(a)$
\item $F_X(x) $ ist nicht abnehmend
\item$ f_x(x) \ge 0 $
 \item $F_X (-\infty) = 0$ und $ F_X(\infty) = \int_{-\infty}^\infty f_X(x) dx =1$


\end{itemize}






\subsection{}
\begin{itemize}
\item
\item
\end{itemize}


















\end{document}